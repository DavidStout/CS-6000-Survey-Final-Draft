\section{Synthesis of Research}\label{sec-compare}
By examining the current research, we see that there is still a need for methods to increase code comprehension. By using current methods of neuroscience \cite{siegmund_measuring_2017, siegmund_understanding_2014, yeh_detecting_2017, nakagawa_quantifying_2014} it is possible to begin efforts to quantify the effectiveness of program comprehension models. As we begin to understand what comprehension model is best suited to understanding source code, we can design better metrics to measure the state of understanding specific attributes of code. Then when newer technologies are created, this process can continue to be refined and will help us to understand the best methodology to use for modeling code comprehension\cite{siegmund_measuring_2017}.

The attributes that lead to a higher rate of code comprehension can then be used to write newer and more robust coding style guides, as well as create automated applications that can evaluate source code ease of comprehension and identify confusing code sections for the refactor and maintenance periods. This will significantly reduce the overhead for developers who currently spend a large segment of their time trying to understand the code. The information can also be useful to researchers that wish to create more robust obfuscation tools, by identifying some key areas that need to be obfuscated.

