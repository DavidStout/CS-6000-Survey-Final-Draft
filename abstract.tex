\begin{abstract}
 \comm{\yanyan{These first few sentences wander kinda off topic. Need to get to the point in abstract.}} \ In
 the realm of Computer Science, where the logic in code can be wildly 
different from one developer to the next, the potential for confusing patterns is abundant. As source code has exploded in individual projects from a few dozen lines to 
hundreds of thousands of lines, and the need for teams of developers has become more commonplace, that possibility of increased
confusion has continued to rise exponentially. This increased potential for confusion has created a higher overhead on  
debugging, re-factoring, and maintenance of projects. Reducing that 
potential for confusion and understanding the underlying cause of confusion is what the field of Source Code Comprehension is all about.
This paper explores the current comprehensional models accepted by the community, and looks at the methods for increasing code interpretation by measuring cognitive understanding in
those program comprehension models, exploring the creation of software clarity metrics, and following coding standards through style guides. In the interest of the completion of source comprehension understanding, we also look at the inversely related field of code obfuscation and how it can be reversed for better code comprehension. By understanding most of the methods that are currently available in code comprehension, it becomes apparent that there is still room for newer methods of identification and solutions to make source code less confusing.
\comm{\yanyan{You could mention something like "there is still room for newer methods of identification and solutions to make source code less confusing".}}\
\end{abstract}
\keywords{Code comprehension, software readability, clarity metrics, style guides, obfuscation}