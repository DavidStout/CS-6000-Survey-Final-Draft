\section{Literature Review}\label{sec-models}
There are varying models of code comprehension that have been
suggested over the years. Most contain some 
common elements, as shown in Figure \ref{fig:models}~\cite{noauthor_pdf_nodate}. 



In this paper we will focus on the three most important:
\begin{itemize}
    \item The Top-down model
    \item The Bottom-up model
    \item The Integrated (combined) model
\end{itemize}

The \textbf{Top-down model} describes a process that consists of applying knowledge about the domain of the
program and then applying this knowledge to the structure of the 
code~\cite{noauthor_pdf_nodate}. This
process of assimilating data is driven by recognizing beacons, or sets of features that typically
indicate the occurrence of certain structures or operations within the code, as described by Brooks~\cite{brooks_towards_1983}. 
Top-down comprehension is a functional understanding of the code, starting
with the big picture and then breaking down the source code from there into smaller segments in order
to understand the whole.

The \textbf{Bottom-up model} describes an assimilation process in 
which programmers start with individual code
statements, and chunk or group these into higher level abstractions. 
The process is then repeated at
successively higher levels until a complete mental representation of the program is formed~\cite{noauthor_pdf_nodate}. 
There have been investigations into the human factors of code
comprehension, showing that minimally small patterns in code can 
and do confuse programmers~\cite{gopstein_understanding_2017, 
gopstein_prevalence_2018}. Generally
seen as a method of chunking microstructures in the code into macrostructures to comprehend the
program flow, Bottom-up modelling uses environmental information to form a perception of the whole by
identifying individual base elements in great detail in order to link them to greater and greater
subsystems. 

Some authors do not accept that there is a dominance of either top-down or bottom-up
comprehension, choosing instead to believe that the developer switches between the two models as the
situation needs. This ideology is made clear in the 
\textbf{Integrated model}, which is a combined model of both the Top-down and Bottom-up approaches \cite{noauthor_pdf_nodate}. As the developer goes through the
assimilation process of comprehending the source code, they switch from a functional understanding of the
program, to a control-flow understanding as is needed for greater overall comprehension. This approach
seems to be a more intuitive process and depends on the developer's previous experience and the berth of
their knowledge base.

By understanding these comprehension models, researchers are able to evaluate the best methods for measuring code comprehension, create software clarity metrics for automated comprehension tools, and develop better coding style guides. We will use these models to help organize and classify the current research.
